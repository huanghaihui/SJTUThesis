%# -*- coding: utf-8-unix -*-
%%==================================================
%% abstract.tex for SJTU Master Thesis
%%==================================================

\begin{abstract}

随着软件领域的快速发展,软件的迭代速度不断变快,源码变更次数激增。然而源代码的变更带来的影响是不可预测的,可能会导致软件出现新的缺陷,引起不可估量的损失,也增加了软件测试与验证的难度。因此如何高效的处理代码变更带来的影响是加速软件开发以及提升软件的可靠性和安全性的重要问题之一。
 
虽然针对源码变更的研究集中于源码级别的分析,但是很多的软件分析与验证是基于中间表达式这个级别的,源码级别的分析显然并不能直接显示需要的影响结果且需要研究者进一步的分析与研究。因此,本文把从源码到中间表达式的翻译过程作为切入点,从而达到基于源码变化能够直接修改原来的中间表达式文件的效果。

LLVM因其优异的表现受到了学术界和工业界的热捧。Clang作为它的前端,在处理性能以及代码结构方面相比GCC也有着较大的优势。因此,我们选择Clang和LLVM这个优势组合作为研究的基础,实现了基于源码变更从而达到准确定位到中间表达式变化的指令,最后生成新的中间表达式文件的目的。

本课题的主要研究内容:

首先,通过语法分析等方法对源码的每个节点进行标记,加入到Clang到中间表达式的翻译过程中,生成带有指定标记的中间表达式。

其次,利用Diff等工具进行源码分析,得到源码的变化信息包括变化节点以及与之其对应的编辑操作等等。

最后,根据LLVM的指令集,先前生成的中间表达式文件以及上一步所得到的源码变化信息,生成新的中间表达式的文件。

实验表明,本文所提出的方法在保证了生成的中间表达式的正确性的基础上,提高了对源码变化带来的影响的处理效率,得到了很好的效果。当然,本文提出的方法还可以在中间代码生成速度以及中间表示文件的生成过程中文件复杂性方面有所改善和提高。

\keywords{\large 中间表达式 \quad 源码变更 \quad LLVM \quad Clang}
\end{abstract}

\begin{englishabstract}

The rapid development of software has made source code change become more and more frequently. However, the impact  is unpredictable, which may lead to software crash, causing incalculable damage. How to deal with the impacts of source code changes is becoming a key issue to accelerate software development and enhance the reliability and security of software.

Although many research work about source code change only focus on source level, many analysis and verifications are based on intermediate representation level. Of course, the entry point is to study how to track source code change and generating intermediate representation. 

LLVM is so popular in academic and industrial for its excellent performance. Clang, as its front-end, also has great advantages in code structure and performance compared to GCC. Therefore, Clang and LLVM are chosen as base tools to locate the changes from source code to intermediate representation.

This paper firstly add flags to the node through syntax analysis and other methods and generate intermediate representations files with specific flags.

Then obtaining the change information including change nodes with corresponding edit operations and so on.

Finally, generating new intermediate representation files based on the change information in the previous step, llvm instructions and the previously unchanged intermediate representation file.

Experimental results show that the method proposed in this paper improves the efficiency of processing the impacts of source code changes based on the correctness of generated intermediate representation files. Certainly, the method should be improved in some aspects of the speed of generating intermediate representation files and simplifying the generated files during the process.

\englishkeywords{\large Intermediate Representation, Source Code Change, LLVM, Clang}
\end{englishabstract}

